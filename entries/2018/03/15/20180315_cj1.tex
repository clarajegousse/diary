% mainfile: ../../../../master.tex
\subsection{SYBR\cR Gold nucleic acid gel stain}
% The part of the label after the colon must match the file name. Otherwise,
% conditional compilation based on task labels does NOT work.
\label{task:20180315_cj1}
\tags{lit,rna,dna}
\authors{cj}
%\files{}
%\persons{}

SYBR\cR Gold is a nucleic acid gel stain. It is very sensitive and allows detection of double- or single-stranded DNA and RNA in electrophoresis gels while our usual stain (SYBR\cR Safe) does not allow detection of RNA and is less sensitive.

However, because SYBR\cR Gold is different from SYBR\cR Safe, I must adapt my protocol. 

Considerations:
\sidenote{\url{https://www.researchgate.net/post/Why_does_SYBR_Gold_interfere_with_1Kb_DNA_ladder2}}
\begin{itemize}
\item It seems that it is preferable to not cast SYBR stains into the gels (which is what we usually do!). Here is what the documentation manual says:
\begin{quote}
\textbf{Smearing and Distorted Bands}: If the dye is cast into the gel, or the nucleic acid prestained during loading, you may see some smearing or distortion of the bands. SYBR Green stains are very sensitive to nucleic acid overloading. We recommend that each lane contain 1–5 ng of nucleic acid per band to avoid this problem. We do not recom- mend including the dye in the running buffer as it will disrupt the migration of the nucleic acids and cause smearing of the bands.
\end{quote}
\sidenote{\url{https://tools.thermofisher.com/content/sfs/manuals/td004.pdf}}
\item allow SYBR\cR Gold to warm up at room tempature before preparing the working solution.
\end{itemize}

