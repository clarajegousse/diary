% mainfile: ../../../../master.tex
\subsection{Océans: une usine chimique qui se dérègle}
% The part of the label after the colon must match the file name. Otherwise,
% conditional compilation based on task labels does NOT work.
\label{task:20180318_cj0}
\tags{lit}
\authors{cj}
%\files{}
\persons{Nicolas Martin,Marilaure Grégoire,Laurent Bopp,Pascal-Jean Lopez}

\comment{Comme souvent dans le bus, j'écoute des podcast et j'ai trouvé l'emission de La Méthode Scientifique du 13 mars 2018 très intéressante. L'émission s'intitule \textit{Océans: une usine chimique qui se dérègle}}.

\sidenote{\url{https://www.franceculture.fr/emissions/la-methode-scientifique/la-methode-scientifique-du-mardi-13-mars-2018}}

Quelle est la chimie propre aux océans et quel rôle joue-t-elle ? Pour quelles raisons cette chimie est-elle perturbée et quelles sont les conséquences de ces perturbations ? Quelles solutions sont aujourd’hui envisagées pour enrayer le phénomène ?

Saviez-vous que l’on trouve dans les océans terrestres tous les éléments chimiques connus, ne serait-ce qu’à l’état de traces parce qu’avant d’être en endroit où faire trempette pour se rafraîchir l’été, l’Océan Mondial est avant tout une incroyable, immense usine physico-chimique, qui brasse les éléments terrestres et atmosphériques dans des processus de conversion qui sont encore pour partie mal compris. Ce que l’on comprend bien en revanche, c’est que l’activité humaine, depuis près de deux siècles, est en train de bouleverser cet équilibre. Résultat : l’océan se réchauffe, s’étouffe, s’acidifie. La dernière fois que cela s’est produit, 90\% des espèces vivantes ont disparu. 

Océans : une usine physico-chimique qui se dérègle : c’est le problème qui va nous occuper pour l’heure qui vient.

Et pour en parler, nous avons le plaisir de recevoir aujourd’hui Marilaure Grégoire, directrice de recherche à l’Université de Liège, vice-président du groupe international de recherche Global Ocean Oxygen Network, son équipe vient de publier début janvier une étude dans la revue Science qui décrit l’ampleur planétaire des processus de désoxygénation des océans et Laurent Bopp, océanographe et climatologue, directeur de recherche CNRS au Laboratoire des Sciences du Climat et de l’Environnement de l’Institut Pierre Simon Laplace.

\subsubsection{Le reportage du jour}

Rencontre avec Pascal-Jean Lopez, chargé de recherche CNRS au sein de l’unité de recherche "Biologie des organismes et des écosystèmes aquatiques" au Muséum National d’Histoire Naturelle. Comment les diatomées, ces microalgues qui jouent un rôle prépondérant dans la photosynthèse océanique peuvent-elles réagir à l’acidification des océans ?