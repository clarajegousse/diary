% mainfile: ../../../../master.tex
\subsection{Effect of Lugol Iodine}
% The part of the label after the colon must match the file name. Otherwise,
% conditional compilation based on task labels does NOT work.
\label{task:20180306_cj1}
\tags{lit}
\authors{cj}
%\files{}
%\persons{}

Many published sampling protocols make use of Lugol's solution. You add this preservative in amounts to achieve a 1\% final concentration (1 part per 100). The iodine in Lugol's is effectively bacteriostatic, but it causes a number of changes in algal cells.\sidenote{Binds to algal cells, but what about micro-algae?} For example, iodine will bind with starch to form a blue-black complex. This reaction is useful in identifying starch, which is present in some algal groups, but not in others. Moreover, Lugol's solution does not preserve cell structure well in many cases, making identification difficult. For later microscopic observation, it is preferable to preserve in glutaraldehyde (which we use for flow cytometry analysis).

Also, \citet{maki2017sample} concluded that "\textit{preserving the samples in acidic Lugol’s solution resulted in equal DNA yields and PCR performance, but affected community profiles.}"

So in my opinion, this is a little concerning ... especially for the 16S and 18S rRNA studies, but also for the metagenomics. 

\comment{In any case, we can't change it and I don't think we could have done better anyways since this survey was organised at the last minute.}

I also read the paper by \citet{williams2016marine}.