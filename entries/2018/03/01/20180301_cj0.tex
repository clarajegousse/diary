% mainfile: ../../../../master.tex
\subsection{Extraction with gentle bead beating and isolation of total DNA and RNA with  AllPrep Mini Kit from micro-algae culture}
% The part of the label after the colon must match the file name. Otherwise,
% conditional compilation based on task labels does NOT work.
\label{task:20180301_cj0}
\tags{lab,dna,rna,extr}
\authors{cj}
%\files{}
%\persons{}

\subsubsection{Introduction}

AllPrep\cR DNA/RNA Kits allows the simultaneous purification of genomic DNA and total RNA from the same cell or tissue sample.

Two days ago, I performed this same experiment with a bead beating step as Viggó suggested. The experiment was a success as I obtained both DNA and RNA but surprisingly, I retrieved more DNA than RNA while I should be able to obtain more RNA than DNA.

There are two steps that could have affected my RNA yield:
\begin{itemize}
\item the bead beating step
\item the fact that I did not totally dry the column before eluting the RNA
\end{itemize}

So this time, I will make sure I dry the column before eluting the RNA and I will use a more gently bead beating step: three times 10 seconds at 30 Hz instead of two times 30 seconds at 30 Hz. Also, Pauline Vannier recommended me to use icy water to cool down my tube instead of ice only. \sidenote{Icy water will cool down the tube efficiently because it increases the surface area of contact with the cold.}

\comment{This time, after the bead beating, I did not forget to take with me my ice bucket with the 70\% Ethanol and the 10 mM Tris-HCl buffer to avoid running up and down the stairs.}

\subsubsection{Sample disruption and homogenisation of cells}

\begin{enumerate}
\item Harvest 2~mL of micro-algae cultures.
\comment{María came to talk to me before I had the time to place my tube in the centrifuge so it remained for about 15 min at room temperature.}
\item Centrifuge for 20 min at maximum speed at 4\degree C.
\item Discard supernatent.
\item Add 600~\uL of Buffer RLT.
\item Add approx. 0.2 mg of beads. \sidenote{Mia prepared them, I must check the diameters of the beads.}
\item Disrupt MixerMill MM400 by Retsch using the program P9 (300 Hz) for 10 seconds three times and immerse tube icy water for 30 sec after each round.
\comment{Next time, I will try to also cool down my tube before the bead beating step.}
\item Centrifuge the lysate for 3 min at maximum speed (20 000 x g) at 4\degree C.
\item Transfer carefully the supernatent to an AllPrep DNA spin column placed in a 2~mL collection tube.
\item Close the lid.
\item Centrifuge for 30 sec. at 8000 x g.
\mistake{I actually centrifuge for 1 min but I don't think it matters}
\item Use the flow-through for RNA purification: proceed to Total RNA purification.
\item Place the AllPrep DNA spin column in a new 2~mL collection tube and keep at room temperature.\comment{I leave my tube on ice.}
\end{enumerate}

\subsubsection{Total RNA purification}

\begin{enumerate}
\item Add 1 vol. of 70\% ethanol to the flow-through collected previously. \comment{Here, one volume is 600~\uL}
\item Mix well by pipetting.
\item Transfer up to 700~uL of the sample to an RNeasy spin column placed in a 2~mL collection tube.
\item Centrifuge for 15 sec. at 8000 x g at 4\degree C.
\item Discard the flow-through.
\item I repeat the three previous steps with the remaining volume of sample to make sure the column is saturated in RNA.
\item Add 700~\uL of Buffer RW1 to the RNeasy spin column.
\item Close the lid.
\item Centrifuge for 15 sec. at 8000 x g at 4\degree C.
\item Discard the flow-thought.
\item Add 500~\uL of Buffer RPE to the RNeasy spin column.
\item Close the lid.
\item Centrifuge for 15 sec. at 8000 x g at 4\degree C.
\item Discard the flow-thought.
\item Add 500~\uL of Buffer RPE to the RNeasy spin column.
\item Close the lid.
\item Centrifuge for 2 min at maximum speed at 4\degree C.
\item Discard the flow-thought.
\comment{At this step, I could have performed the optional step to dry the column by centrifuging at maximum speed for one min.}
\item Place the RNeasy spin column in a new 1.5~mL collection tube.
\item Add 50~\uL of RNase-free water directly to the spin column membrane. 
\item Close the lid gently.
\item Centrifuge for 1 min at 8000 x g to elute the RNA.
\comment{I repeated the elution of the RNA with an extra 30~\uL of RNase-free water, which makes a final volume of 80~\uL of RNase-free water containing the RNA.}
\end{enumerate}

\subsubsection{Genomic DNA purification}

\begin{enumerate}
\item Add 500~\uL of Buffer AW1 to the AllPrep DNA spin column from the Lysis.
\item Close the lid.
\item Centrifuge for 15 sec. at 8000 x g at 4\degree C to wash the column membrane.
\item Discard the flow-thought.
\item Add 500~\uL of Buffer AW2 to the AllPrep DNA spin column
\item Centrifuge for 2 min at maximum speed (20 000 x g) at 4\degree C to wash the column membrane.
\item Place the AllPrep DNA spin column in a new 1.5~mL collection tube.
\item Add 100~\uL of Tris-HCl buffer (pH 8) to the DNA AllPrep spin column membrane and close the lid.
\item Incubate at room temperature for 1 min.
\item Centrifuge for 1 min at 8000 x g to elute the DNA.
\mistake{I forgot to incubate for one minute the first time! lucky I actually repeat this step.}
\comment{I added an extra 100~\uL of Tris-HCl to the column, incubate 1 min and centrifuge at 8000 x g for 1 min to make sure all DNA was eluted.}
\end{enumerate}

\subsubsection{Conclusion}

This procedure takes me 2 hours to complete (from a little after 10:00 AM to a little before 12:00 AM) even though I was interepted few times.
