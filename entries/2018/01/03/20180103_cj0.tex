% mainfile: ../../../../master.tex
\subsection{Extraction of total DNA and RNA from marine samples}
% The part of the label after the colon must match the file name. Otherwise,
% conditional compilation based on task labels does NOT work.
\label{task:20180103_cj0}
\tags{rna,dna,lab,extr}
\authors{cj}
%\files{}
%\persons{}

\subsubsection{Introduction}

As a training for myself, I will try to simultaneously extract the RNA and DNA from Mia's culture (she suspect that this one contains cyanobacteria) \texttt{20170623 SB5 0m F/2}.
I will follow the protocol adapted from \citet{schneider2017extraction} and \citet{chomczynski2006single}.

\subsubsection{Lysis and homogenisation}

\begin{enumerate}
\item 2 mL of cultures split in 2 2mL-Eppendorf tubes.
\item Centrifuge at 9300 RCF (10000 rpm) at 4\degree C for 5 min (Eppendorf 5415R).
\comment{I can see a light green pellet.} 
\item Remove the supernatent.
\item Add 2 mL of denaturing solution.
\item Vortex.
\item Incubate at room temperature for 10 min.
\item Split lysate in 6 tubes (each containing approximately 0.6 mL).
\end{enumerate}

\subsubsection{Simultaneous DNA and RNA extraction}
\begin{enumerate}
\item Add 0.4 mL of 2~M sodium acetate pH 4.
\mistake{I added too much, it was supposed to be 0.1 vol so I should not have added 0.4mL of sodium acetate buffer to each tube, but 60 \uL in each eppendorf tube.}
\item Add 1 volume of phenol pH 4.
\item Mix thoroughly by inversion.
\item Add 0.2 volume of chloroform:isoamyl alcohol (24:1).
\item Cool on ice for 15 min.
\comment{I believe it remained on ice for longer than 15 min}
\item Centrifuge for 20 min at 7200 g at 4\degree C (during the centrifugation, the mixture should separate into a lower phenol phase and interphase and an upper aqueous phase).
\comment{After centrifugation, I am unable to see the two phases, so I combined all the contents of the 6 eppendorf tubes into one 15 mL falcon tube, and I added 1mL of chloroform isoamyl alcohol and I repeat the centrifugation step with a different rotor so at 5000 RPM.}
\item Transfer carefully the supernatent (aqueous phase) which contains mostly RNA to a clean Falcon tube.
\comment{Here I am able to transfer easely 4 mL of supernatent.}
\item Add 2~mL of 2~M sodium acetate buffer pH 4.
\item Mix thoroughly by inversion.
\item Centrifuge for 20 min at 4\degree C.
\mistake{Once again, it seems that the 2 mL of sodium acetate buffer pH 4 did not migrate to the upper phase, so I add an extra 1 mL of chloroform isoamyl alcohol and I repeat the centrifugation step.}
\item Transfer carefully the upper phase to the already collected aqueous phase.
\comment{The final volume is about 7 mL.}
\item For later DNA isolation, add 1 volume of Tris-base pH 10.5 to the phenolic phase and mix well.
\comment{To avoid the same issue as last time, I check the pH with pH paper to ensure the pH is back to neutral/alkaline which lead me to have two falcon tubes each containing 10 mL.}
\item Continue with RNA precipitation. 
\end{enumerate}

\subsubsection{RNA isolation}
\begin{enumerate}
\item Add 1 volume of chloroform:isoamyl alcohol (24:1).
\item Incubate at -20\degree C for at least 15 min (but no longer than 1h).
\item Centrifuge again for 20 min at 4\degree C to separate the two phases.
\item Transfer the aqueous upper plase to a frish tube
\item Repeat the chloroform extraction by adding 2~mL of 2~M sodium acetate buffer to the organic phase, mix and centrifuge again, and finally transfer the supernatent to the already collected aqueous phase.
\item Add 1/500 volume of glycogen (20 mg/mL) \sidenote{The glycogen solution was prepared by Stephen back in 2014 and kept at -20\degree C.}
\item Mix vigorously.
\item Add 1 volume of isopropanol to precipitate RNA.
\item Incubate samples at -20\degree C overnight. 
\comment{By incubating overnight it will precipitate everything so it is very likely that I will also precipitate some contaminants. But at least I am sure all the RNA will be precipitated and I can just perform more washing steps afterwards.}
\item Continue with DNA isolation.
\end{enumerate}

\subsubsection{DNA isolation}
\begin{enumerate}
\item Mix tube containing the extracted DNA obtained from previous steps.
\item Centrifuge again for 20 min at 5000 RPM at 4\degree C to separate the two phases.
\item Transfer the upper aqueous phase into fresh tubes (at this point, check pH again with pH paper).
\item Add 2~mL of 1~M Tris-base pH 10.5 to the lower phenolic phase and repeat mixing and centrifugation to separate the two phases.
\mistake{I added the citrate buffer pH 4 instead of the Tris-base, so I had to add a lot of 1~M Tris-base pH 10.5 so that the pH is back to neutral (check with pH paper) so now, I have a lot of tubes ...}
\item Transfer the upper aqueous phase to the already collected aqueous phase (at this point, check pH again with pH paper).
\item Add 1 volume of chloroform:isoamyl alcohol (24:1) to the aqueous phase.
\item Mix thoroughly by inversion.
\item Centrifuge for 10 min at 4\degree C to separate the two phases/
\item Add 1/500 volume of glycogen (20 mg/mL).
\item Mix thoroughly.
\item Add 2.5 volumes of ice-cold pure ethanol to precipitate the DNA. 
\item Mix thoroughly.
\item Incubate samples at -20\degree C overnight.
\sidenote{I decided to do the incubations overnight, because I am not feeling very well today (periods, yeah!), so it is better that I do the concentration of the nucleic acids tomorrow.}
\comment{By incubating overnight it will precipitate everything so it is very likely that I will also precipitate some contaminants. But at least I am sure all the DNA will be precipitated and I can just perform more washing steps afterwards.}
\end{enumerate}




