% mainfile: ../../../../master.tex
\subsection{Extraction with bead beating and isolation of total DNA and RNA with  AllPrep Mini Kit from micro-algae culture}
% The part of the label after the colon must match the file name. Otherwise,
% conditional compilation based on task labels does NOT work.
\label{task:20180227_cj0}
\tags{lab,dna,rna,extr}
\authors{cj}
%\files{}
%\persons{}

\subsubsection{Introduction}
AllPrep\cR DNA/RNA Kits allows the simultaneous purification of genomic DNA and total RNA from the same cell or tissue sample.

After discussing with Viggó last week, he recommended me to try this method with a bead beating step and to compare the quality of the DNA on the gel and see if the bead beating actually shears the DNA a lot.

\subsubsection{Sample disruption and homogenisation of cells}

\begin{enumerate}
\item Harvest 2~mL of micro-algae cultures.
\item Centrifuge for 20 min at maximum speed (20 000 x g) at 4\degree C.
\item Discard supernatent.
\item Add 600~\uL of Buffer RLT.
\item Add approx. 0.2 mg of beads. \sidenote{Mia prepared them, I must check the diameters of the beads.}
\item Disrupt MixerMill MM400 by Retsch using the program P9 (300 Hz) for 30 seconds twice and rest sample on ice for 1 min after each round.
\item Centrifuge the lysate for 3 min at maximum speed (20 000 x g) at 4\degree C.
\item Transfer carefully the supernatent to an AllPrep DNA spin column placed in a 2~mL collection tube.
\item Close the lid.
\item Centrifuge for 30 sec. at 8000 x g.
\item Use the flow-through for RNA purification: proceed to Total RNA purification.
\item Place the AllPrep DNA spin column in a new 2~mL collection tube and keep at room temperature.\comment{I leave my tube on ice.}
\end{enumerate}

\subsubsection{Total RNA purification}

\begin{enumerate}
\item Add 1 vol. of 70\% ethanol to the flow-through collected previously. \comment{Here, one volume is 600~\uL}
\item Mix well by pipetting.
\item Transfer up to 700~uL of the sample to an RNeasy spin column placed in a 2~mL collection tube.
\item Centrifuge for 15 sec. at 8000 x g at 4\degree C.
\item Discard the flow-through.
\item I repeat the three previous steps with the remaining volume of sample to make sure the column is saturated in RNA.
\item Add 700~\uL of Buffer RW1 to the RNeasy spin column.
\item Close the lid.
\item Centrifuge for 15 sec. at 8000 x g at 4\degree C.
\item Discard the flow-thought.
\item Add 500~\uL of Buffer RPE to the RNeasy spin column.
\item Close the lid.
\item Centrifuge for 15 sec. at 8000 x g at 4\degree C.
\item Discard the flow-thought.
\item Add 500~\uL of Buffer RPE to the RNeasy spin column.
\item Close the lid.
\item Centrifuge for 2 min at maximum speed at 4\degree C.
\item Discard the flow-thought.
\comment{At this step, I could have performed the optional step to dry the column by centrifuging at maximum speed for one min.}
\item Place the RNeasy spin column in a new 1.5~mL collection tube.
\item Add 50~\uL of RNase-free water directly to the spin column membrane. 
\item Close the lid gently.
\item Centrifuge for 1 min at 8000 x g to elute the RNA.
\comment{I repeated the elution of the RNA with an extra 30~\uL of RNase-free water, which makes a final volume of 80~\uL of RNase-free water containing the RNA.}
\end{enumerate}

\subsubsection{Genomic DNA purification}

\begin{enumerate}
\item Add 500~\uL of Buffer AW1 to the AllPrep DNA spin column from the Lysis.
\item Close the lid.
\item Centrifuge for 15 sec. at 8000 x g at 4\degree C to wash the column membrane.
\item Discard the flow-thought.
\item Add 500~\uL of Buffer AW2 to the AllPrep DNA spin column
\item Centrifuge for 2 min at maximum speed (20 000 x g) at 4\degree C to wash the column membrane.
\item Place the AllPrep DNA spin column in a new 1.5~mL collection tube.
\item Add 100~\uL of Tris-HCl buffer (pH 8) to the DNA AllPrep spin column membrane and close the lid.
\item Incubate at room temperature for 1 min.
\item Centrifuge for 1 min at 8000 x g to elute the DNA.
\comment{I added an extra 100~\uL of Tris-HCl to the column, incubate 1 min and centrifuge at 8000 x g for 1 min to make sure all DNA was eluted.}
\end{enumerate}

\subsubsection{Conclusion}

It took me less than 2 hours to finish the full procedure (from a little after 10:00 AM to a little before 12:00 AM) which means that it should be totally possible for me to complete this procedure for all my samples in one morning and quantify the nucleic acids in the following afternoon.

The important modifications I have made compared to last time: my centrifuge steps were a stronger (but according to the protocol, I actually had made mistakes last time in setting the centrifuge). Also I used bead beating during the homogenisation of the sample and I did not do the \textit{optional} step of drying the RNeasy spin column.



