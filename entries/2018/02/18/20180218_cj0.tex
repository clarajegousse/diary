% mainfile: ../../../../master.tex
\subsection{Amplification by PCR with OneTaq\cR polymerase and \texttt{EUK1} primers of DNA (and cDNA) extracted from micro-algae cultures with two different kits}
% The part of the label after the colon must match the file name. Otherwise,
% conditional compilation based on task labels does NOT work.
\label{task:20180218_cj0}
\tags{lab,pcr,dna,cdna}
\authors{cj}
%\files{}
%\persons{}

\subsubsection{Introduction}
Today, I will basically repeat the same experiment as yesterday. I will try to amplify the DNA and cDNA extracted from micro-algae cultures two different kits: the MasterPure\texttrademark~ by Epicentre and the AllPrep DNA/RNA Mini Kit by Qiagen. The only difference is that I will try to amplify these samples with eukaryote primers \texttt{EUK1}.


\subsubsection{Description of the DNA templates}
\begin{enumerate}
\item \texttt{{\color{Grass} CJ20180201\_MP}}: DNA extracted from 2~mL micro-algae culture with modified MasterPure\texttrademark~ kit.
\item \texttt{{\color{Aqua} CJ20180207\_AP}}: DNA extracted from 2~mL micro-algae culture with AllPrep kit.
\item \texttt{{\color{Lavender} CJ20180213\_AP}}: DNA extracted from 4~mL micro-algae culture with AllPrep kit.
\item \texttt{{\color{Sunflower} CJ20180214\_MP}}: DNA extracted from 4~mL micro-algae culture with modified MasterPure\texttrademark~ kit.
\item \texttt{CJ20180217\_MP1+}: cDNA obtained by reverse transcription \sidenote{Reverse transcription achieve with ProtoScript First Strand Synthesis by New England Biolab.} from RNA extracted from 2~mL micro-algae culture with modified MasterPure\texttrademark~ kit. 
\item \texttt{CJ20180217\_AP1+}: cDNA obtained by reverse transcription from RNA extracted from 2~mL micro-algae culture with AllPrep kit. 
\item \texttt{CJ20180217\_AP2+}: cDNA obtained by reverse transcription from RNA extracted from 4~mL micro-algae culture with AllPrep kit. 
\item \texttt{CJ20180217\_MP2+}: cDNA obtained by reverse transcription from RNA extracted from 4~mL micro-algae culture with modified MasterPure\texttrademark~ kit. 
\end{enumerate}

\subsubsection{Description of controls}
My controls are:
\begin{itemize}
\item[+] diluted DNA from \textit{Thermoanaerobacterium sp.} (Bacteria)
\item[+] DNA from \textit{Thermococcus barophilus} 1 ng/\uL (Archeae)
\item[-] DNA from Cod 1 ng/\uL (Eukaryote)
\item[-] Autoclaved MilliQ water
\end{itemize}

\important{I did not have do use archaeal and bacterial DNA, but I am so used to defrost them before doing a PCR, that I did it anyways. So I decided to use them which will allow me to make sure these DNA are not amplified.}


\subsubsection{Description of the polymerase}

I will use the \href{https://www.neb.com/products/m0481-onetaq-hot-start-dna-polymerase#Product%20Information}{OneTaq\cR Hot Start DNA polymerase} by New England BioLabs with the \gls{emp} primers.

\subsubsection{Description of the primers}

\texttt{EUK1} is the primer set corresponding to the V4 region of the 18S rRNA gene. It is the 3NDf and V4\_euk\_R2 published by \citet{braate2010freshwater}.


\subsubsection{Preliminary setup}

\begin{enumerate}
\item Defrost all reagents and samples on ice (1h). \sidenote{It is very important that reagents (especially primers and dNTPs) are fully defrosted otherwise the concentration will not be accurate.}
\item Place all required material in LabCair \gls{pcr} Workstation and turn on \gls{uv} light for at least 20 min.
\comment{I forgot to place gloves under \gls{uv} lights. Too bad, I will just sterilise my gloves with ethanol and use few drops of RNase Zap.}
\end{enumerate}

\subsubsection{Methods}

\begin{enumerate}
\item Work in LabCair \gls{pcr} Workstation
\item Multiply the volume of each reagent by the number of individual \gls{pcr} reactions you wish to perform (including the positive and negative controls) and add 2 extra to account for pipetting error (see table \ref{tab:20180218_mastermix}).
\comment{I have 8 samples, 4 controls, and 2 extra, which means I must prepare a master mix for 16 reactions.}
\item In a single 1.5mL-Eppendorf tube combine the following:
	\begin{itemize}
	\item Sterile dH2O
	\item 5X OneTaq Reaction Buffer
	\item OneTaq GC Enhancers
	\item dNTP mix (10 mM each nt)
	\item primers (EUK1 primers)
	\item OneTaq\cR Hot Start DNA Polymerase
	\end{itemize}
\item Mix the contents by gently pipetting up and down several times (keep tube on ice)
\item Transfer  22.5 \textmu L of Master Mix into small \gls{pcr} tubes
\item add 10 \textmu L of \gls{dna} template in each \gls{pcr} tube (on the pre-PCR bench) following the layout shown in figure \ref{tikz:20180218_pcr_racks}.
\item Secure the tops to the \gls{pcr} tubes
\item Tap gently tubes to mix up everything
\item Centrifuge briefly to bring all the liquid to the bottom before placing it in the \gls{pcr} machine
\end{enumerate}
\comment{I finished the tube of OneTaq Hot Start Polymerase (lucky I ordered some recently).}

\begin{table}[htbp]
\caption{Master Mix}
\label{tab:20180218_mastermix}
\centering
\begin{tabular}{l r r c}
\toprule
Primers & \multicolumn{2}{c}{EUK1}\\
\cmidrule(l){2-3}
 & \multicolumn{2}{c}{Volumes (\uL) for} \\
 \cmidrule(l){2-3}
 & 1 reaction & 16 reactions \\ 
\midrule 
dH2O & 12.80~\uL & 204.80~\uL\\
5X OneTaq Reaction Buffer & ~5.00~\uL & ~80.00~\uL \\
GC Enhancers & ~0.50~\uL & ~40.00~\uL \\
dNTPs (10mM) & ~0.50~\uL & ~~8.00~\uL \\
Forward primer (10\textmu M) & ~0.75~\uL & ~12.00~\uL \\
Reverse primer (10\textmu M) & ~0.75~\uL & ~12.00~\uL \\
OneTaq\cR DNA Polymerase & ~0.20\uL & ~~3.20\uL \\
\midrule
Template DNA & 2.5\uL & - \\
\midrule
$V_{f}$ (\uL) & 25 & - \\
\bottomrule
\end{tabular}
% }
\end{table}


\begin{figure}[htbp]
\caption{Samples organisation in PCR tubes}
\label{tikz:20180218_pcr_racks}
\newcommand*\smp[1]{\texttt{#1}}
\resizebox{\textwidth}{!}{
\begin{tikzpicture}[%remember picture,
  well/.style={rounded rectangle, minimum width=3.5cm, fill=lightgrey, thick, inner sep=3pt, node distance=3pt, minimum height=0.5cm},
  line/.style={fill=white, thick, inner sep=3pt, node distance=3pt, align=left},
  racknum/.style={thick, inner sep=3pt, node distance=3pt, minimum width=0.6cm},
  rack/.style={fill=Aqua, thick, inner sep=3pt, node distance=3pt},
  primer/.style={fill=white, align=left}
  ]
    \node[primer] (emp) {\textbf{EUK1}};
      \node[rack, fill=white, below=of emp] (labels1) {
        \begin{tikzpicture}
          \node [racknum, draw=white] (1)  {};
          \node [well, fill=white, right=of 1] (well1)  {A};
          \node [well, fill=white, right=of well1] (well2)  {B};
          \node [well, fill=white, right=of well2] (well3)  {C};
          \node [well, fill=white, right=of well3] (well4)  {D};
          \node [well, fill=white, right=of well4] (well5)  {E};
          \node [well, fill=white, right=of well5] (well6)  {F};
          \node [well, fill=white, right=of well6] (well7)  {G};
          \node [well, fill=white, right=of well7] (well8)  {H};
        \end{tikzpicture}
      };
      \node[rack, below=of labels1] (rack1) {
        \begin{tikzpicture}
          \node [racknum] (1)  {\#6~};
          \node [well, right=of 1] (well1)  {\smp{-ARCH}};
          \node [well, right=of well1] (well2)  {\smp{-BACT}};
          \node [well, right=of well2] (well3)  {\smp{+COD}};
          \node [well, right=of well3] (well4)  {\smp{-H2O}};
          \node [well, right=of well4] (well5)  {\smp{MP2-}};
          \node [well, right=of well5] (well6)  {\smp{AP2-}};
          \node [well, right=of well6] (well7)  {\smp{MP1-}};
          \node [well, right=of well7] (well8)  {\smp{}};
        \end{tikzpicture}
      };
      \node[rack, below=of rack1] (rack2) {
        \begin{tikzpicture}
          \node [racknum] (1)  {\#2~};
          \node [well, right=of 1] (well1)  {\smp{MP1}};
          \node [well, right=of well1] (well2)  {\smp{AP1}};
          \node [well, right=of well2] (well3)  {\smp{AP2}};
          \node [well, right=of well3] (well4)  {\smp{MP2}};
          \node [well, right=of well4] (well5)  {\smp{MP1+}};
          \node [well, right=of well5] (well6)  {\smp{AP1+}};
          \node [well, right=of well6] (well7)  {\smp{AP2+}};
          \node [well, right=of well7] (well8)  {\smp{MP2+}};
        \end{tikzpicture}
      };
\end{tikzpicture}
}

\end{figure}

\comment{Figure \ref{tikz:20180218_pcr_racks} shows the actual layout of the samples in the rack, but I must admit this layout is not the most logical. In a way, rack \#6 contains all controls while rack \#2 contains reactions mix with DNA and cDNA templates.}

% \begin{table}[htbp]
% \caption{Thermocycler settings}
% \label{tab:20180218_thermocycler_settings}
% \centering
% \begin{tabular}{l r r}
%  & \multicolumn{2}{c}{\texttt{EMP}}\\
% \cmidrule(l){2-3}
% Steps & T(\degree C) & Time \\
% \midrule
% Initial denaturation & 94 & 30 sec \\
% \midrule
% Denaturation & 94 & 30 sec \\
% Annealing & 68 & 40 sec \\
% Extension & 68 & 20 sec \\
% \midrule
% Final extension & 68 & 2 min \\
% Infinite hold & 4 & - \\
% \bottomrule
% \end{tabular}
% % }
% \end{table}

I used the preset method on the thermocycler.

\important{I increased the number of cycles from 30 to 35 cycles.}



