% mainfile: ../../../../master.tex
\subsection{Ph.D. project overview}
% The part of the label after the colon must match the file name. Otherwise,
% conditional compilation based on task labels does NOT work.
\label{task:20180209_cj0}
\tags{mime,meet,phd}
\authors{cj}
%\files{}
\persons{Stephen Knobloch}

\begin{itemize}
\item[\texttt{Q1}] What seems - to you - the most critical part of the project?
\comment{The extraction of the nucleic acids.}
\item[\texttt{Q2}] What seems - to you - the most challenging part of the project?
\comment{The analysis of the data.}
\item[\texttt{Q3}] How would you integrate flow cytometry data analysis into the story for my thesis?
\item[\texttt{Q4}] How to make it more bioinformatic-oriented?
\comment{Start with few samples to sent sequencing, so I can start working on the analysis very quickly and see how it turns out before doin all together.}
\item[\texttt{Q5}] If nucleic acid yields are to low, would you rather focus on metagenomics or on metatranscriptomics?
\comment{Ask if it is possible for some companies to do metagenomic sequencing from a small DNA quantities.}
\item[\texttt{Q6}] What - do you think - is the most time-consumming?
\comment{The analysis of the data.}
\end{itemize}

\comment{Stephen also recommends to insure first enough data so that I can complete my Ph.D. but I should not be following the research proposal.}

He said that with 30 Gb if data, if I have a species representing 1\% of my environment, I will get a coverage of 600 which is more than enough for assembling a genome (knowing that the most abundant marine species have a draft genome).

He reckons that my first draft should be with the publication of the metagenomes on their own, the second draft could be the comparison of these metagenomes, and then I will see.