% mainfile: ../../../../master.tex
\subsection{Extraction of total DNA and RNA with AllPrep mini kit}
% The part of the label after the colon must match the file name. Otherwise,
% conditional compilation based on task labels does NOT work.
\label{task:20180207_cj0}
\tags{lab,dna,rna,extr}
\authors{cj}
%\files{}
%\persons{}

\subsubsection{Introduction}

The AllPrep DNA/RNA Mini Kit (cat. no. \texttt{80204}) should be stored dry at room temperature (15–25\degree C) and is stable for at least 9 months under these conditions.
\mistake{This kit is expired since 2011, so if it does not work, it could be because of this.}

The AllPrep DNA/RNA Mini Kit purifies genomic DNA and total RNA simultaneously from a single sample. Lysate from homogenized cells or tissue is first passed through an AllPrep DNA spin column to isolate DNA, then through an RNeasy\cR spin column to isolate RNA.

\subsubsection{Sample disruption and homogenization of cells}

I work with the micro algae cultures from Mia. 
\begin{enumerate}
\item Transfer 2~mL of culture to a clean Eppendorf tube.
\item Centriguge for 30s at 5000 g to pellet the cells.
\item Discard supernatent.
\item Add 600~\uL of Buffer RLT Plus
\item Vortex
\item Centrifuge 3 min at 12000 x g
\item Transfer the lysate to an AllPrep DNA Sprin column placed in a 2 mL collection tube (supplied). Close the lid gently, and centrifuge for 30 sec at >= 8000 x g (>= 10000 rpm)..
\item Use the flow through for RNA purification (see following section).
\item Place the AllPrep DNA Sprin column in a new 2 mL collection tube (supplied). Store at room temperature (15-25\degree C) or at 4\degree C for later DNA purification. 
\comment{Do not store the column at room temperature or at 4\degree C for long periods. DO NOT FREEZE THE COLUMNS.}
\end{enumerate}

\subsubsection{Total RNA purification}
\begin{enumerate}
\item Add 1 volume of 70\% ethanol to the flow-through from step 2 in Quick- StartProtocol AllPrep DNA/RNA Mini Kit, Part 1. Mix well by pipetting. Do not centrifuge. Proceed immediately to step 2.
\item Transfer up to 700 μl of the sample, including any precipitate, to an RNeasy spin column placed in a 2 ml collection tube (supplied). Centrifuge for 15 s at >= 8000 x g (>= 10,000 rpm). Discard the flow-through.
\comment{Reuse this collection tube through steps 3, 4, and 5.}
\item Add 700 μl Buffer RW1 to the RNeasy spin column. Close the lid, and
centrifuge for 15 s at >= 8000 x g (>= 10,000 rpm). Discard the flow-through.
\item Add 500 μl Buffer RPE to the RNeasy spin column. Close the lid, and
centrifuge for 15 s at >=  x g (>= 10,000 rpm). Discard the flow-through.
\item Add 500 μl Buffer RPE to the RNeasy spin column. Close the lid, and centrifuge for 2 min at >= 8000 x g ( >=10,000 rpm).
Optional: Place the RNeasy spin column in a new 2 ml collection tube (supplied). Discard the old collection tube with the flow-through. 
\item Centrifuge at full speed for 1 min to dry the spin column membrane.
Place the RNeasy spin column in a new 1.5 ml collection tube (supplied). \item Add 30–50 μl RNase-free water directly to the spin column membrane. Close the lid gently, and centrifuge for 1 min at >=8000 x g ( >=10,000 rpm) to elute the RNA.
\comment{I elute the RNA with 50~\uL of RNase-free water because I usually resuspend my nucleic acid pellet in 50~\uL of buffer when performing nucleic acid extraction.}
Optional: If the expected RNA yield is >30 μg, repeat step 6 using another 30–50 μl of RNase-free water, or using the eluate from step 6 (if high RNA concentration is required). Reuse the collection tube from step 6.
\end{enumerate}

\subsubsection{Genomic DNA purification}
\begin{enumerate}
\item Add 500 μl Buffer AW1 to the AllPrep DNA spin column (in 2 ml collection tube) from step 4 in Quick-StartProtocol AllPrep DNA/RNA Mini Kit, Part 1. Close the lid gently, and centrifuge for 15 s at >=8000 x g ( >= 10,000 rpm) to wash the spin column membrane. Discard the flow-through. Reuse the collection tube in step 2.
\item  Add 500 μl Buffer AW2 to the AllPrep DNA spin column. Close the lid gently, and centrifuge for 2 min at full speed to wash the spin column membrane.
\item  Place the AllPrep DNA spin column in a new 1.5 ml collection tube (supplied). Add 100 μl Buffer EB directly to the spin column membrane and close the lid. Incubate at room temperature (15–25°C) for 1 min. Centrifuge for 1 min at >= 8000x g (>=10,000 rpm) to elute the DNA.
\comment{There was no more EB buffer in the kit, so I use my homemade 10 mM Tris-HCl buffer ph8 instead.}
Optional: Repeat step 3 using another 100 \uL of Buffer EB, or using the eluate from step 3 (if higher DNA concentration is required). Reuse the collection tube from step 3.
\end{enumerate}

