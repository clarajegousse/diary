% preamble.tex
\documentclass[10pt, a4paper]{report}

% LANGUAGES 

% to use all caracters including accents
\usepackage[utf8]{inputenc}
\usepackage[english]{babel}
\usepackage[euler]{textgreek} % for greek letters (e.g. \textalpha)
\usepackage{textcomp} % for the \text{\textperthousand}
% Use the palatine font
%\usepackage[sc]{mathpazo}
%\linespread{1.05} % Palatino needs more leading (space between lines)
% Choose the font encoding
\usepackage[T1]{fontenc}
\usepackage{gensymb} % for the \degree

% SKIP LINES BETWEEN PARAGRAPHS
\usepackage[parfill]{parskip}

% COLOR SETTINGS

% load a colour package
\usepackage[usenames,dvipsnames]{xcolor}

% define colors
\definecolor{aaublue}{rgb}{0.00,0.41,0.66}% dark blue
\definecolor{aaublue}{RGB}{33,26,82}% dark blue
\definecolor{aaugray}{RGB}{84,97,110}% gray blue
\definecolor{aaulblue}{RGB}{194,193,204} % light blue
\definecolor{GriffithRed}{HTML}{DC143C}% griffith red
\definecolor{MatisBlue}{HTML}{003478}% Blue Matís

% flattastic color palette BY erigon)
\definecolor{grapefruit}{HTML}{ED5565}
\definecolor{Grapefruit}{HTML}{DA4453}
\definecolor{bittersweet}{HTML}{FC6E51}
\definecolor{Bittersweet}{HTML}{E9573F}
\definecolor{Carrot}{HTML}{E67E22}
\definecolor{sunflower}{HTML}{FFCE54}
\definecolor{Sunflower}{HTML}{F6BB42}
\definecolor{grass}{HTML}{A0D468}
\definecolor{Grass}{HTML}{8CC152}
\definecolor{mint}{HTML}{48CFAD}
\definecolor{Mint}{HTML}{37BC9B}
\definecolor{aqua}{HTML}{4FC1E9}
\definecolor{Aqua}{HTML}{3BAFDA}
\definecolor{jeans}{HTML}{5D9CEC}
\definecolor{Jeans}{HTML}{4A89DC}
\definecolor{lavender}{HTML}{AC92EC}
\definecolor{Lavender}{HTML}{967ADC}
\definecolor{rose}{HTML}{EC87C0}
\definecolor{Rose}{HTML}{D770AD}
\definecolor{lightgrey}{HTML}{F5F7FA}
\definecolor{LightGrey}{HTML}{E6E9ED}
\definecolor{mediumgrey}{HTML}{CCD1D9}
\definecolor{MediumGrey}{HTML}{AAB2BD}
\definecolor{darkgrey}{HTML}{656D78}
\definecolor{DarkGrey}{HTML}{434A54}

% GRAPHICS AND TABLES

\usepackage{longtable}
\usepackage{tabularx}
\usepackage{pdflscape}

% The standard graphics inclusion package
\usepackage{graphicx}
\DeclareGraphicsExtensions{% set the priority
    .pdf,.PDF,%
    .png,.PNG,%
    .jpg,.mps,.jpeg,.jbig2,.jb2,.JPG,.JPEG,.JBIG2,.JB2}

\usepackage{tikz}
% Create graphics using PGF/TikZ
\usepackage{tikz,pgfplots}
\usetikzlibrary{matrix,backgrounds,shapes,arrows,fadings}
\usetikzlibrary{external}
\tikzexternalize[
  mode=convert with system call,% or graphics if exists, no graphics, only graphics
  figure list=true]
\tikzexternaldisable

% Set up how figure and table captions are displayed
\usepackage{caption}
\captionsetup{%
  font=footnotesize,% set font size to footnotesize
  labelfont=bf % bold label (e.g., Figure 3.2) font
}
\usepackage{subcaption}
% Make the standard latex tables look so much better
\captionsetup[sub]{%
  font=scriptsize,% set font size to footnotesize
}

\usepackage{gantt}
\usepackage{longtable}

\usepackage{array,booktabs}
\usepackage{multirow}
\usepackage{multicol}

% Enable the use of frames around, e.g., theorems
% The framed package is used in the example environment
\usepackage{framed}
%adobe reader fix when using transparency in tikz
\pdfpageattr{/Group <</S /Transparency /I true /CS /DeviceRGB>>}

% MATHEMATICS

\usepackage{amsmath} % for the equation env
\usepackage{amssymb} % Adds new math symbols
% Use theorems in your document
% The ntheorem package is also used for the example environment
% When using thmmarks, amsmath must be an option as well. Otherwise \eqref doesn't work anymore.
\usepackage[framed,amsmath,thmmarks]{ntheorem}
\usepackage[version=3]{mhchem} % for chemistry

% LAYOUT

% Change margins, papersize, etc of the document
\usepackage[
  inner=18mm,% left margin on an odd page
  outer=45mm,% right margin on an odd page
  marginparwidth=35mm, % width of the marginal paragraph
  headheight=14pt,
  ]{geometry}

% Modify how \chapter, \section, etc. look
% The titlesec package is very configureable
\usepackage{titlesec}
%\titleformat{\chapter}[display]{\normalfont\bfseries}{}{0pt}{\Large}
\newcommand{\hsp}{\hspace{20pt}}
\titleformat{\chapter}[hang]{\Huge\bfseries}{\thechapter\hsp\textcolor{LightGrey}{|}\hsp}{0pt}{\Huge\bfseries}

\titleformat*{\section}{\normalfont\Large\bfseries\color{black}}
\titleformat*{\subsection}{\normalfont\large\bfseries\color{black}}


%\titleformat*{\subsubsection}{\normalfont\normalsize\bfseries\color{aaublue}}
%\titleformat*{\paragraph}{\normalfont\normalsize\bfseries\color{aaublue}}
%\titleformat*{\subparagraph}{\normalfont\normalsize\bfseries\color{aaublue}}

% Change the headers and footers
\usepackage{fancyhdr}
\pagestyle{fancy}
\fancyhf{} %delete everything
\renewcommand{\headrulewidth}{0pt} %remove the horizontal line in the header
\fancyhead[RE]{\color{MediumGrey}\small\nouppercase\leftmark} %even page - chapter title
\fancyhead[LO]{\color{MediumGrey}\small\nouppercase\rightmark} %uneven page - section title
\fancyhead[LE,RO]{\thepage} %page number on all pages
% Do not stretch the content of a page. Instead,
% insert white space at the bottom of the page
\raggedbottom

% Enable arithmetics with length. Useful when
% typesetting the layout.
\usepackage{calc}

% fix the marginpar command so it is always on the correct side
\usepackage{mparhack}

% truncate a long file name
\usepackage[breakall]{truncate}

% ALGO, PSEUDO CODE AND CODE

\usepackage{amsmath}

% package for writing algorithms and pseudocode
\usepackage{algpseudocode}

% change the appearance of comments
\algrenewcommand{\algorithmiccomment}[2][\hfill]{#1{\color{aaugray}\# #2}}

% float environment for algorithms
\usepackage{algorithm}

% display source code
\usepackage{listings}
\lstset{
%  tabsize=2,
	inputencoding=utf8,
	extendedchars=true
%	basicstyle=\footnotesize\fontfamily{Courier},
	extendedchars=true,
	basicstyle=\footnotesize\ttfamily,
	breaklines=true,
	numbers=left, 
	numberstyle=\tiny, 
	stepnumber=1,
	backgroundcolor=\color{white},
	commentstyle=\color{MediumGrey},
	keywordstyle=\color{Grass},
	stringstyle=\color{Aqua},
	frame=lines,}

\lstdefinestyle{couleurs}{
	inputencoding=utf8,
	extendedchars=true
%	basicstyle=\footnotesize\fontfamily{Courier},
	extendedchars=true,
	basicstyle=\footnotesize\ttfamily,
	breaklines=true,
	numbers=left, 
	numberstyle=\tiny, 
	stepnumber=1,
	backgroundcolor=\color{white},
	frame=lines,
  	moredelim=**[is][\color{Grapefruit}]{@Grapefruit}{@Grapefruit},
  	moredelim=**[is][\color{Bittersweet}]{@Bittersweet}{@Bittersweet},
  	moredelim=**[is][\color{Sunflower}]{@Sunflower}{@Sunflower},
  	moredelim=**[is][\color{Grass}]{@Grass}{@Grass},
  	moredelim=**[is][\color{Mint}]{@Mint}{@Mint},
  	moredelim=**[is][\color{Jeans}]{@Jeans}{@Jeans},
}
  
\lstset{literate=%
{æ}{{\ae}}1
{å}{{\aa}}1
{ø}{{\o}}1
{Æ}{{\AE}}1
{Å}{{\AA}}1
{Ø}{{\O}}1
{á}{{\'a}}1
{à}{{\`a}}1
{á}{{\^a}}1
{é}{{\'e}}1
{è}{{\`e}}1
{ê}{{\^e}}1
{ù}{{\`u}}1
{û}{{\^u}}1
{î}{{\^i}}1
}

% BIBLIOGRAPHY
%\usepackage[backend=bibtex,bibencoding=utf8]{biblatex}
%\addbibresource{bib/bib2017.bib}

\usepackage[round,comma]{natbib}
\bibliographystyle{plainnat}

% MISC

% Add the indices
\usepackage[splitindex]{imakeidx}
\indexsetup{toclevel=part}
\makeindex[title=People Index, name=person, intoc]
\makeindex[title=File Index, name=file, intoc, columns=1]
\makeindex[title=Author Index, name=author, intoc]
\makeindex[title=Tag Index, name=tag, intoc]

\usepackage[
%  disable, %turn off todonotes
  colorinlistoftodos, %enable a coloured square in the list of todos
  textwidth=\marginparwidth, %set the width of the todonotes
  textsize=scriptsize, %size of the text in the todonotes
  ]{todonotes}

% LINKS AND HYPERREF

% Enable hyperlinks and insert info into the pdf
% file. Hypperref should be loaded as one of the 
% last packages
\usepackage{hyperref}
\hypersetup{%
	pdfpagelabels=true,%
	plainpages=false,%
	pdfauthor={},%
	pdftitle={},%
	pdfsubject={},%
	bookmarksnumbered=true,%
	colorlinks,%
	citecolor=black,%
	filecolor=black,%
	linkcolor=black,% you should probably change this to black before printing
	urlcolor=black,%
	pdfstartview=FitH%
}

% SET SANS SERIF FONT
\renewcommand{\familydefault}{\sfdefault}
\renewcommand*\sfdefault{phv}

% ABBREVIATIONS
\usepackage[acronym]{glossaries}
\usepackage{acro}
\makeglossaries
% abbr.tex
% abbreviations

% names
\newacronym{mime}{MIME}{Microbes in the Icelandic Marine Environment}
\newacronym{emp}{EMP}{Earth Microbiom Project}
\newacronym{osd}{OSD}{Ocean Sampling Day}
\newacronym{ebi}{EBI}{European Bioinformatics Institute}
\newacronym{ena}{ENA}{European Nucleotide Archive}
\newacronym{kic}{KIC}{Knowledge and Innovation Community}
\newacronym{eit}{EIT}{European Institute of Innovation \& Technology}

% file formats
\newacronym{csv}{CSV}{comma separated values} 
\newacronym{tsv}{TSV}{tabulation separated values} 
\newacronym{fcs}{FCS}{flow cytometry standards} 

% chemicals and molecule names
\newacronym{dna}{DNA}{deoxyribonucleic acid} 
\newacronym{rna}{RNA}{ribonucleic acid} 
\newacronym{mrna}{mRNA}{messenger-RNA}
\newacronym{rrna}{rRNA}{ribosomal-RNA}
\newacronym{trna}{tRNA}{transfer-RNA}
\newacronym{cdna}{cDNA}{coding-DNA}
\newacronym{cscl}{CsCl}{cesium chloride}
\newacronym{ddh2o}{dd\ce{H2O}}{distilled and deionised water}
\newacronym{ctab}{CTAB}{cetyl trimethylammonium bromide}
\newacronym{sds}{SDS}{sodium dodecyl sulfate}
\newacronym{edta}{EDTA}{ethylenediaminetetraacetic acid}
\newacronym{pci}{PCI}{phenol:chloroform:isoamyl alcohol (25:24:1)}
\newacronym{dmso}{DMSO}{dimethyl sulfoxide}
\newacronym{peg}{PEG}{polyethylene glycol}
\newacronym{depc}{DEPC}{diethyl pyrocarbonate}
\newacronym{tae}{TAE}{Tris-acetate-EDTA}


% machine and devices
\newacronym{eps}{EPS}{electrophoresis power supply}
\newacronym{ccd}{CCD}{charge coupled device}

% techniques
\newacronym{sip}{SIP}{stable-isotope probing} 
\newacronym{ngs}{NGS}{new generation sequencing}
\newacronym{pcr}{PCR}{polymerase chain reaction}
\newacronym{rtpcr}{RT-PCR}{reverse transcription polymerase chain reaction}

% partner abbreviations
\newacronym{sbr}{SBR}{Station Biologique de Roscoff} 
\newacronym{mri}{MRI}{Marine Research Institute}

% misc
\newacronym{uv}{UV}{ultra violet}
\newacronym{pbs}{PBS}{Portable Batch System}




% makeglossaries outline

% NEW ENVRIONMENTS

% Numbered environment
\newcounter{rquestion}[section]
\newenvironment{rquestion}[1][]{\refstepcounter{rquestion}\par\medskip
   \noindent \textbf{Research question~\therquestion. #1} \rmfamily}{\medskip}

\newcommand{\uL}{~\textmu L }
\newcommand{\cR}{\textsuperscript \textregistered}

\newcommand{\sidenote}{\todo[backgroundcolor=white, bordercolor=white,noline]}
\newcommand{\mistake}{\todo[backgroundcolor=grapefruit, bordercolor=grapefruit,inline]}
\newcommand{\important}{\todo[backgroundcolor=sunflower, bordercolor=sunflower,inline]}
\newcommand{\comment}{\todo[backgroundcolor=white, bordercolor=white,inline]}


